\documentclass[11pt, halfparskip]{article}

\usepackage{tikz}
\usepackage{hyperref}
%\usepackage{amsmath}
% \usepackage{mathtools}


\begin{document}
\noindent This example assumes that the layers are numbered from 1 to n, with the first layer being the first hidden layer and the nth layer being the output layer.
It also assumes the costfunction $\sum \frac{1}{2}(target - out)^2$ and that the sigmoid function (sig(x)=$\frac{1}{1 + e^{-x}}$) is used as an activation function and
uses the given example of a 2-4-3-2 network.\\
The terminology and naming of neurons outputs etc. is mostly from this amazing \href{https://mattmazur.com/2015/03/17/a-step-by-step-backpropagation-example}{blogpost}
and is a prerequisite for understanding the following calculations.

\noindent\\
Now if you want to calculate the gradient for a weight of the outputlayer, for example $w_{31}$ connecting neurons $c_1$ and $d_1$, this is fairly simple. Weight
$w_{31}$ is right in the beginning (or rather at the end of the network, but since we are trying to do backpropagation the end is our beginning), most of the network
becomes irrelevant when calculating the (partial) derivative.\\
Since out intention is to minimize the error we will need to calculate the derivative of the error function $E_{total}$ with respect to $w_{31}$. When following the chain rule to calculate 
the derivative we get:
    \begin{equation}
	\frac{\partial E_{total}}{\partial w_{31}} = \frac{\partial E_{total}}{\partial out_{d1}} * \frac{\partial out_{d1}}{\partial net_{d1}} * \frac{\partial net_{d1}}{\partial w_{31}}
    \end{equation}
    
\noindent Because I had some trouble understanding how the left hand side of the equation equals the right hand side at first and this is essential to understanding anything following this
equation, I will explain it in more detail before proceeding.\\
For that let's split up the Error function into its components. In this particular example we begin with $E_{total} = \sum \frac{1}{2}(target-out)^2$. Having only two output neurons
this equals $E_{total} = \frac{1}{2}(target_{d_1} - out_{d_1})^2 + \frac{1}{2}*(target_{d_2} - out_{d_2})^2$.
The first part of the sum is the error $E_{d_1}$ for neuron $d_1$ and the second is the error $E_{d_2}$ for neuron $d_2$.\\
We want to derive with respect to $w_{31}$ and not the output of some neuron so lets split those up again. Then we would get $out_{d_1} = sig(net_{d_1})$ and 
$out_{d_2} = sig(net_{d_2})$.\\
To have $w_{31}$ in our equation we need to split it up once more and get $net_{d_1} = w_{31}*out_{c_1} + w_{xx}*out_{c_2} + w_{xx}*out_{c_3}$ and
$net_{d_2} = w_{xx}*out_{c_1} + w_{xx}*out_{c_2} + w_{xx}*out_{c_3}$. I have omitted stating the correct indices for anything but $w_{31}$ because, as we will see soon,
they become irrelevant when deriving with respect to $w_{31}$.\\

\noindent\\
Putting it back together we get
    \begin{equation}
    	E_{d_1} = \frac{1}{2}(target_{d_1} - sig(w_{31}*out_{c_1} + w_{xx}*out_{c_2} + w_{xx}*out_{c_3}))^2
    \end{equation}
When beginning to derive this formula, we already need the chain rule. We derive the outer function and multiply it by the derivative of the inner function resulting in
    \begin{equation}%WICHTIG: Explain splitting the Error in two derivatives
    	(inner)*\frac{\partial inner}{\partial w_{31}} \textrm{, mit } inner = target_{d_1} - sig(w_{31}*out_{c_1} + w_{xx}*out_{c_2} + w_{xx}*out_{c_3})
    \end{equation}
Now to get the first factor you could also derive the equation
   \begin{center}
$E_{d_1} = \frac{1}{2}(target_{d_1} - out_{d_1})^2$
   \end{center}
with respect to $out_{d_1}$, which would be
    \begin{center}
	$\frac{\partial E_{d_1}}{\partial out_{d_1}} = \frac{\partial \frac{1}{2}(target_{d_1}-out_{d_1})^2}{\partial out_{d_1}} = (target_{d_1}-out_{d_1})*(-1)$
    \end{center}
The only difference is that $out_{d_1}$ isn't split up into its components and the $-1$ at the end. The $-1$ is a leftover from deriving the inner function, which looks like it
doesn't exist in equation (3), but it is only hidden in $\frac{\partial inner}{\partial w_{31}}$.\\
Now as you can see, $\frac{\partial E_{d_1}}{\partial out_{d_1}}$ is the first factor in equation one, so we already got that part figured out. To get the second and third factor we now
have to continue deriving with $\frac{\partial inner}{\partial w_{31}}$ (see equation 3). Since $w_{31}$ is still nested in a function, in this case sig(x), we need to apply the chain
rule a second time, which results in
    \begin{equation}
    	\frac{\partial inner }{\partial w_{31}} = -sig(net_{d_1})*(1 - sig(net_{d_1}))*\frac{\partial net_{d_1}}{\partial w_{31}},
    \end{equation}
    \begin{center}
    	mit $net_{d_1}=w_{31}*out_{c_1} + w_{xx}*out_{c_2} + w_{xx}*out_{c_3}$
    \end{center}
Let's look at the first factor first, which ist pretty much only the derivative or the sigmoid function. Now since the output $out_{d_1}$ is nothing but $sig(net_{d_1})$, the derivative of
$out_{d_1}$ is also the derivative of $sig(net_{d_1})$. Therefore we get
    \begin{equation}
	\frac{\partial out_{d_1}}{\partial w_{31}} = \frac{\partial sig(net_{d_1})}{\partial w_{31}}
    \end{equation}
As seen in equation (4), when we apply the chainrule to $sig(net_{d_1})$ we first derive with respect to $net_{d_1}$ and then multiply by the derivative with respect to $w_{31}$.
This means we now have the equation
    \begin{equation}
	\frac{\partial sig(net_{d_1})}{\partial w_{31}} = \frac{\partial sig(net_{d_1})}{\partial net_{d_1}}*\frac{\partial net_{d_1}}{\partial w_{31}}
    \end{equation}
From the equality in equation (5) we can deduct that
    \begin{equation}
    	\frac{\partial sig(net_{d_1)}}{\partial w_{31}} = \frac{\partial sig(net_{d_1})}{\partial net_{d_1}}*\frac{\partial net_{d_1}}{\partial w_{31}} = \frac{\partial out_{d_1}}{\partial 			net_{d_1}}*\frac{\partial net_{d_1}}{\partial w_{31}} = \frac{\partial out_{d_1}}{\partial w_{31}}
    \end{equation}
All those steps put back together we have now proven, that by following the chain rule $\frac{\partial E_{d_1}}{\partial w_{31}} = \frac{\partial E_{total}}{\partial out_{d1}} * \frac{\partial out_{d1}}{\partial net_{d1}} * \frac{\partial net_{d1}}{\partial w_{31}}$ (see equation 1).\\
While we wanted it for $E_{total}$ and not just $E_{d_1}$ you can see, when looking at $E_{d_1}$, that it does not contain $w_{31}$ at all, meaning that deriving with respect to $w_{31}$ must result in 0. \\
In other words, for weight $w_{31}$ applies that
    \begin{equation}
    	\frac{\partial E_{d_1}}{\partial w_{31}} = \frac{\partial E_{total}}{\partial w_{31}} = \frac{\partial E_{total}}{\partial out_{d1}} * \frac{\partial out_{d1}}{\partial net_{d1}} * 			\frac{\partial net_{d1}}{\partial w_{31}}
    \end{equation}

\end{document}